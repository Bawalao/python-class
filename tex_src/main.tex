
\documentclass[a4paper,11pt]{article}

\usepackage[T1]{fontenc}
\usepackage[utf8]{inputenc}
\usepackage{graphicx}
\usepackage{xcolor}
\usepackage{dirtree}

\renewcommand\familydefault{\sfdefault}
\usepackage{tgheros}

\usepackage{amsmath,amssymb,amsthm,textcomp}
\usepackage{enumerate}
\usepackage{multicol}
\usepackage{tikz}

\usepackage{geometry}
\geometry{left=25mm,right=25mm,%
bindingoffset=0mm, top=20mm,bottom=20mm}


\linespread{1.3}

\newcommand{\linia}{\rule{\linewidth}{0.5pt}}

% custom theorems if needed
\newtheoremstyle{mytheor}
    {1ex}{1ex}{\normalfont}{0pt}{\scshape}{.}{1ex}
    {{\thmname{#1 }}{\thmnumber{#2}}{\thmnote{ (#3)}}}

\theoremstyle{mytheor}
\newtheorem{defi}{Definition}

% my own titles
\makeatletter
\renewcommand{\maketitle}{
\begin{center}
\vspace{2ex}
{\huge \textsc{\@title}}
\vspace{1ex}
\\
\linia\\
\@author \hfill \@date
\vspace{4ex}
\end{center}
}
\makeatother
%%%

% custom footers and headers
\usepackage{fancyhdr}
\pagestyle{fancy}
\lhead{}
\chead{}
\rhead{}
\lfoot{Learning Python with Doudou}
\cfoot{}
\rfoot{Page \thepage}
\renewcommand{\headrulewidth}{0pt}
\renewcommand{\footrulewidth}{0pt}
%

% code listing settings
\usepackage{listings}
\lstset{
    language=Python,
    basicstyle=\ttfamily\small,
    aboveskip={1.0\baselineskip},
    belowskip={1.0\baselineskip},
    columns=fixed,
    extendedchars=true,
    breaklines=true,
    tabsize=4,
    prebreak=\raisebox{0ex}[0ex][0ex]{\ensuremath{\hookleftarrow}},
    frame=lines,
    showtabs=false,
    showspaces=false,
    showstringspaces=false,
    keywordstyle=\color[rgb]{0.627,0.126,0.941},
    commentstyle=\color[rgb]{0.133,0.545,0.133},
    stringstyle=\color[rgb]{01,0,0},
    numbers=left,
    numberstyle=\small,
    stepnumber=1,
    numbersep=10pt,
    captionpos=t,
    escapeinside={\%*}{*)}
}

%%%----------%%%----------%%%----------%%%----------%%%

\begin{document}

\title{Learning Python with Doudou}

\author{Brian Teixeira, Doudou University}

\date{05/25/2020}

\maketitle

Welcome to this Python class! In this series of tutorials, you will
learn how to fully master the python programming language through
explanations, examples and exercises. This python class will be split
into three main parts: Basics, Intermediate and Advanced. Following
this tutorial, you will be able to unlock a new series of tutorials
called Machine Learning with Python! Pretty exciting right?

\vspace{5mm}

The ``Basics`` will mostly present general stuff about the python
programming language. You will get to learn about functions,
variables, scopes, packages etc. Some of the things will sound very
basic and obvious, but it's always great to take a fresh look at things.
I will try to go as far as I can while describing the behavior of the
language, especially with regards to the management of the memory.

\vspace{5mm}

The ``Intermediate`` part will be the longest part of this tutorial,
not because it's the most complex, but because it's the one that
will cover the biggest part of the language, because as you will understand,
the python language, while rich, is not complex. In this section, you will
get to learn about objects, exceptions, generators and other stuff. I will
insist a lot on the object oriented programming part, as this is, in my
opinion, a very important aspect of the language, and of programming in general.
Don't make me say everything should always be object, but in Python it's kind
of the case. I love objects, but only when things are well designed. Thus,
in this part I will go a bit further than just focusing on the language itself
and will go through some good design practices.

\vspace{5mm}

Finally, the ``Advanced`` part will go through some more rarely used features
of the language, including multiprocessing or networking. You will quickly
realize that this part is not called advanced because it's difficult, but rather
because it's about things you will be less likely to use in your everyday
python journey.

\vspace{5mm}

For each point of the lessons, you will find definitions, examples and
exercises.  For the first lessons, the exercises will be very fast and
academic, but the more you progress in this series of tutorials, the
more they will turn into projects.  There is no recommended time to
finish exercises or projects, because we do not code against the
clock. I will create repositories on github where you will be able
to push the code. Just let me know when you are done with an exercise
or a project, and I will correct it!

\vspace{5mm}

All right, enough talking, let's go!

\newpage

\tableofcontents

\newpage


\section{Basics}

Welcome to the first part of the Doudou Python tutorial!  This part
will cover all basic Python primitives including functions, variables,
how they are managed etc. Let's go!

\subsection{Variables}

Python is a (mostly) interpreted dynamically typed language. Why
mostly? Because it is possible to compile Python. We will get into
more details about it in the advanced section. Why dynamically typed?
Because in Python, you don't need to explicitly specify the type of a
variable, it will be deduced based on the value you are assigning the
variable to. Moreover, you can reassign the same variable with a
different type, because the type will dynamically change based on the
value you are assigning to it. Let's look at an example. In Python,
we can use the ``type'' keyword to evaluate the type of a variable.
Also, we will use the ``print'' built-in function, which takes an
``infinite'' number of arguments as input and displays them on the
standard output, separated by spaces.

\begin{lstlisting}[language=python]
  a = 5
  print(a, 'of type', type(a))
  a = 'hello'
  print(a, 'of type', type(a))
\end{lstlisting}

Now let's execute this code.

\begin{lstlisting}[language=bash]
  doudoush:~$ python3 variable.py
   5 of type <class 'int'>
   hello of type <class 'str'>
  doudoush:~$
\end{lstlisting}

While other languages (like C++) would have complained that you cannot assign
the value ``hello'' of type string to the variable ``a'' of type integer, Python
doesn't care and simply reassign the value and the type of the variable.

\vspace{5mm}
Also, remember that as the language is interpreted, you can execute some code before
an error happens. For example, if you forget to define a variable, the error will
be raised only when you try to use this variable/

\begin{lstlisting}[language=python]
  blossom = 'commander and the leader'
  bubbles = 'the joy and a laugher'

  print('Blossom', blossom)
  print('Bubbles', bubbles)
  print('Buttercup', buttercup)
\end{lstlisting}

Gives the output:

\begin{lstlisting}[language=bash]
  doudoush:~$ python3 variables.py
   Blossom commander and the leader
   Bubbles the joy and a laugher
   Traceback (most recent call last):
     File "variables.py", line 12, in <module>
       print('Buttercup', buttercup)
   NameError: name 'buttercup' is not defined
  doudoush:~$
\end{lstlisting}

Other languages like C++ would have complained that the variable ``buttercup''
did not exist before printing the two other variables.

\subsection{Combining variables}

Another cool aspect of the Python language is the way you can combine variables.
For example, as with many other languages, you can combine high level types with
lower level types. For example, if you do the sum of an integer and a float,
the result will be a variable of type float:

\begin{lstlisting}[language=python]
cactus = 4
cactus += 0.5

print(cactus, 'of type', type(cactus))
\end{lstlisting}

Outputs:

\begin{lstlisting}[language=bash]
  doudoush:~$ python3 variables.py
    4.5 of type <class 'float'>
  doudoush:~$
\end{lstlisting}

Pretty obvious yes. But what happens if we try to add an integer to a string?

\begin{lstlisting}[language=python]
  cactus = 3
  cactus += ' is your lucky number'
\end{lstlisting}

\begin{lstlisting}[language=bash]
  doudoush:~$ python3 variables.py
    Traceback (most recent call last):
    File "variables.py", line 2, in <module>
      cactus += ' is your lucky number'
      TypeError: unsupported operand type(s) for +=: 'int' and 'str'
  doudoush:~$
\end{lstlisting}

This is because it tries to call the ``+'' operator from the ``int'' primitive type.
However, this operator is not defined for variables of type ``str''. We will get
into more details in the following section about functions.

\vspace{5mm}

You may thus wonder how can we combine strings and integers? Well, there are
several ways. The quick and dirty way is to explicitly cast the variable to
the target type. Of course, it does not make sense to try to cast a string to
an integer, as there is no obvious semantic to this operation. So we will do
the other way around, and try to convert 3 into ``3``. The cast operators are simply
the type of the variable used as a function:

\begin{lstlisting}[language=python]
  cactus = 3
  cactus = str(cactus) + ' is your lucky number'
  print(cactus)
\end{lstlisting}

Which works as expected:

\begin{lstlisting}[language=bash]
  doudoush:~$ python3 variables.py
    3 is your lucky number
  doudoush:~$
\end{lstlisting}

However, for string there is a much better way of doing this: string formatting!
There are three different ways of doing string formatting: the old way, the modern way
and the very modern way. The old way is a C-style formatting:

\begin{lstlisting}[language=python]
  cactus = 3
  my_str = '%d is your lucky number' % cactus
\end{lstlisting}

You can also define the way you want the number will be displayed. For example, if you
want to add zeros before the number, you can write ``\%03d'' to specify that
the number should have 3 digits:

\begin{lstlisting}[language=python]
  my_str = '%03d is your lucky number' % 3
  print(my_str)
  my_str = '%03d is our lucky number' % 25
  print(my_str)
\end{lstlisting}

Will output:
\begin{lstlisting}[language=bash]
  doudoush:~$ python3 variables.py
    003 is your lucky number
    025 is our lucky number
  doudoush:~$
\end{lstlisting}

The number got padded with two zeros, and the number twenty five with only one zero,
as it's already a two-digits number.

\vspace{5mm}

Here, the ``\%d'' is the syntax to substitute an entry to an integer. For floating
number, we would use ``\%f'', for string ``\%s''. We can also use this syntax to
substitute multiple variables:

\begin{lstlisting}[language=python]
  cactus = 3
  kind = 'lucky number'
  my_str = '%d is your %s' % (cactus, kind)
\end{lstlisting}


The modern way uses the ``format'' keyword. The main advantage of this method is
that you do not need to specify the type of the variable you want to combine.
We can then write the previous code like this:

\begin{lstlisting}[language=python]
  cactus = 3
  kind = 'lucky number'
  my_str = '{} is your {}'.format(cactus, kind)
\end{lstlisting}

Here, we just need to add ``{}'' every time we need to insert a variable. We them
substitute all of these variables as arguments to the format function. We can also
do zeros padding this way:

\begin{lstlisting}[language=python]
  cactus = 3
  kind = 'lucky number'
  my_str = '{:03} is your {}'.format(cactus, kind)
\end{lstlisting}

Finally, the very modern (Python >= 3.6) way does not require the use of the format
function, and uses a new language syntax:

\begin{lstlisting}[language=python]
  cactus = 3
  kind = 'lucky number'
  my_str = f'{cactus:03} is your {kind}'
  print(my_str)
\end{lstlisting}

Here, we just need to add the ``f'' character before the string to specify that
this string needs formatting, and we then just add the name of the variable directly
where we want them to appear. This of course gives the same output:

\begin{lstlisting}[language=bash]
  doudoush:~$ python3 variables.py
    003 is your lucky number
  doudoush:~$
\end{lstlisting}

\subsection{Functions}

\subsection{Conditions and loop}

\subsection{Exercises}

\subsection{Modules and Packages}

\subsection{Dealing with external resources}
\subsubsection{Parsing arguments}
\subsubsection{Files}
\subsubsection{The sys module}

\subsection{Exercises}

\subsection{Data structures}
\subsubsection{Lists}
\subsubsection{Tuples}
\subsubsection{Sets}
\subsubsection{Dicts}

\subsection{Arguments and named arguments}

\subsection{Project: myFind}


\end{document}

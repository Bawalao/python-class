
\section{Basics}

Welcome to the first part of the Doudou Python tutorial!  This part
will cover all basic Python primitives including functions, variables,
how they are managed etc. Let's go!

\subsection{Variables}

Python is a (mostly) interpreted dynamically typed language. Why
mostly? Because it is possible to compile Python. We will get into
more details about it in the advanced section. Why dynamically typed?
Because in Python, you don't need to explicitly specify the type of a
variable, it will be deduced based on the value you are assigning the
variable to. Moreover, you can reassign the same variable with a
different type, because the type will dynamically change based on the
value you are assigning to it. Let's look at an example. In Python,
we can use the ``type'' keyword to evaluate the type of a variable.
Also, we will use the ``print'' built-in function, which takes an
``infinite'' number of arguments as input and displays them on the
standard output, separated by spaces.

\begin{lstlisting}[language=python]
  a = 5
  print(a, 'of type', type(a))
  a = 'hello'
  print(a, 'of type', type(a))
\end{lstlisting}

Now let's execute this code.

\begin{lstlisting}[language=bash]
  doudoush:~$ python3 variable.py
    5 of type <class 'int'>
    hello of type <class 'str'>
  doudoush:~$
\end{lstlisting}

While other languages (like C++) would have complained that you cannot assign
the value ``hello'' of type string to the variable ``a'' of type integer, Python
doesn't care and simply reassign the value and the type of the variable.

\vspace{5mm}
Also, remember that as the language is interpreted, you can execute some code before
an error happens. For example, if you forget to define a variable, the error will
be raised only when you try to use this variable/

\begin{lstlisting}[language=python]
blossom = 'commander and the leader'
bubbles = 'the joy and a laugher'

print('Blossom', blossom)
print('Bubbles', bubbles)
print('Buttercup', buttercup)
\end{lstlisting}

Gives the output:

\begin{lstlisting}[language=bash]
doudoush:~$ python3 variable.py
   Blossom commander and the leader
   Bubbles the joy and a laugher
   Traceback (most recent call last):
     File "variables.py", line 12, in <module>
       print('Buttercup', buttercup)
   NameError: name 'buttercup' is not defined
doudoush:~$
\end{lstlisting}

Other languages like C++ would have complained that the variable ``buttercup''
did not exist before printing the two other variables.

\subsection{Variables formatting}

\subsection{Functions}

\subsection{Conditions and loop}

\subsection{Exercises}

\subsection{Modules and Packages}

\subsection{Dealing with external resources}
\subsubsection{Parsing arguments}
\subsubsection{Files}
\subsubsection{The sys module}

\subsection{Exercises}

\subsection{Data structures}
\subsubsection{Lists}
\subsubsection{Tuples}
\subsubsection{Sets}
\subsubsection{Dicts}

\subsection{Arguments and named arguments}

\subsection{Project: myFind}
